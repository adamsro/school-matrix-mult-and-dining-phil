% Robert Adams	02/07/2012	CS 311

\documentclass[letterpaper,10pt]{article} %twocolumn titlepage 
\usepackage{graphicx}
\usepackage{amssymb}
\usepackage{amsmath}
\usepackage{amsthm}

\usepackage{alltt}
\usepackage{float}
\usepackage{color}
\usepackage{url}

\usepackage{balance}
\usepackage[TABBOTCAP, tight]{subfigure}
\usepackage{enumitem}
\usepackage{pstricks, pst-node}


\usepackage{geometry}
\geometry{margin=1in, textheight=8.5in} %textwidth=6in

%random comment

\newcommand{\cred}[1]{{\color{red}#1}}
\newcommand{\cblue}[1]{{\color{blue}#1}}

\usepackage{hyperref}

\def\name{Robert Adams}

%% The following metadata will show up in the PDF properties
\hypersetup{
		colorlinks = true,
		urlcolor = black,
		pdfauthor = {\name},
		pdfkeywords = {cs411``operating systems},
		pdftitle = {CS411: Assignment 1, Threading},
		pdfsubject = {CS 411 Assignment 1},
		pdfpagemode = UseNone
}

\begin{document}
\title{CS411: Assignment 1, Threading}
\author{Robert Adams}
\maketitle


\section{What do you think the main point of this assignment is? }


I believe the assignment is designed to give us an introduction to threading.
The matrix multiplication problem was assigned to show us the potential
speedups of threading when there is no synchronization required on an 
n\verb|^|2 problem. The dining philosophers problem was assigned to introduce 
us to mutexes and  show us the possible complexities of thread synchronization. 


\section{How did you approach the problem? Design decisions, algorithm, etc.}


The dining philosophers problem was entirely too complex to create my own solution.
I used code designed in a very similar fashion to the solution presented in the
book. Of note is that hungry philosophers block in the take\_fork function until
a philosopher finished eating runs test(LEFT) and test(RIGHT). This prevents 
hungry philosophers from needlessly spinning and wasting system resources.


For the matrix multiplication problem I found the most efficient way to divide 
the work was by columns. Total columns are divided by the number of threads and
each thread takes responsibility for its section. The last thread will handle 
any odd additional columns. This approach allowed me to use any number of threads
for any number of columns.


To fill the arrays random numbers I used the boost libraries Mersenne Twister
generator seeded with the systems microseconds. This generator should give me
a uniform distribution up to 623 dimensions, not that I really understand 
that, but I know it’s good.


I wished to make my file input function as flexible as possible so I chose
to use C++’s \verb|<<| operator which parses a file and returns the next word using 
spaces, tabs, or line breaks, as separators. If EOF is reached the remaining
spots in the matrixes are filled with zeros.

\section{How did you ensure your solution was correct? Testing details, for instance.}


To test my matrix multiplication algorithm I first ran it for smaller matrices
with fixed values, manually checking the result on paper to ensure the multiplication
is correct. To test that my algorithm is splitting work correctly among threads I
used print statements which may be seen with the verbose, -v, flag, which print
the range each thread is about to calculate. I used gdb to step through my seria
l implementation of matrix multiplication.


To test the dining philosophers problem I stepped through it with a debugger
using many breakpoints to catch switches between threads.


\section{What did you learn?}


I learned that threading can, in certain cases, cause significant speedups.
I also saw that multiple threads which must share limited resources must be 
synchronized, and if done incorrectly, can deadlock, can take more resources
then a serial algorithm, or may end up effectively serialized after
attempting to add synchronization. 

\section{Test Machine System Specs}

Intel(R) Core(TM) i5-2415M CPU @ 2.30GHz 

Number of Processors: 1

Total Number of Cores: 2

L2 Cache (per Core): 256 KB

L3 Cache: 3 MB

Memory: 4 GB

OSX 10.7


\begin{figure}[p]
		\centering
		% GNUPLOT: LaTeX picture
\setlength{\unitlength}{0.240900pt}
\ifx\plotpoint\undefined\newsavebox{\plotpoint}\fi
\sbox{\plotpoint}{\rule[-0.200pt]{0.400pt}{0.400pt}}%
\begin{picture}(1500,900)(0,0)
\sbox{\plotpoint}{\rule[-0.200pt]{0.400pt}{0.400pt}}%
\put(231.0,131.0){\rule[-0.200pt]{4.818pt}{0.400pt}}
\put(211,131){\makebox(0,0)[r]{ 0}}
\put(1419.0,131.0){\rule[-0.200pt]{4.818pt}{0.400pt}}
\put(231.0,212.0){\rule[-0.200pt]{4.818pt}{0.400pt}}
\put(211,212){\makebox(0,0)[r]{ 20}}
\put(1419.0,212.0){\rule[-0.200pt]{4.818pt}{0.400pt}}
\put(231.0,293.0){\rule[-0.200pt]{4.818pt}{0.400pt}}
\put(211,293){\makebox(0,0)[r]{ 40}}
\put(1419.0,293.0){\rule[-0.200pt]{4.818pt}{0.400pt}}
\put(231.0,374.0){\rule[-0.200pt]{4.818pt}{0.400pt}}
\put(211,374){\makebox(0,0)[r]{ 60}}
\put(1419.0,374.0){\rule[-0.200pt]{4.818pt}{0.400pt}}
\put(231.0,455.0){\rule[-0.200pt]{4.818pt}{0.400pt}}
\put(211,455){\makebox(0,0)[r]{ 80}}
\put(1419.0,455.0){\rule[-0.200pt]{4.818pt}{0.400pt}}
\put(231.0,535.0){\rule[-0.200pt]{4.818pt}{0.400pt}}
\put(211,535){\makebox(0,0)[r]{ 100}}
\put(1419.0,535.0){\rule[-0.200pt]{4.818pt}{0.400pt}}
\put(231.0,616.0){\rule[-0.200pt]{4.818pt}{0.400pt}}
\put(211,616){\makebox(0,0)[r]{ 120}}
\put(1419.0,616.0){\rule[-0.200pt]{4.818pt}{0.400pt}}
\put(231.0,697.0){\rule[-0.200pt]{4.818pt}{0.400pt}}
\put(211,697){\makebox(0,0)[r]{ 140}}
\put(1419.0,697.0){\rule[-0.200pt]{4.818pt}{0.400pt}}
\put(231.0,778.0){\rule[-0.200pt]{4.818pt}{0.400pt}}
\put(211,778){\makebox(0,0)[r]{ 160}}
\put(1419.0,778.0){\rule[-0.200pt]{4.818pt}{0.400pt}}
\put(231.0,859.0){\rule[-0.200pt]{4.818pt}{0.400pt}}
\put(211,859){\makebox(0,0)[r]{ 180}}
\put(1419.0,859.0){\rule[-0.200pt]{4.818pt}{0.400pt}}
\put(231.0,131.0){\rule[-0.200pt]{0.400pt}{4.818pt}}
\put(231,90){\makebox(0,0){ 0}}
\put(231.0,839.0){\rule[-0.200pt]{0.400pt}{4.818pt}}
\put(432.0,131.0){\rule[-0.200pt]{0.400pt}{4.818pt}}
\put(432,90){\makebox(0,0){ 500}}
\put(432.0,839.0){\rule[-0.200pt]{0.400pt}{4.818pt}}
\put(634.0,131.0){\rule[-0.200pt]{0.400pt}{4.818pt}}
\put(634,90){\makebox(0,0){ 1000}}
\put(634.0,839.0){\rule[-0.200pt]{0.400pt}{4.818pt}}
\put(835.0,131.0){\rule[-0.200pt]{0.400pt}{4.818pt}}
\put(835,90){\makebox(0,0){ 1500}}
\put(835.0,839.0){\rule[-0.200pt]{0.400pt}{4.818pt}}
\put(1036.0,131.0){\rule[-0.200pt]{0.400pt}{4.818pt}}
\put(1036,90){\makebox(0,0){ 2000}}
\put(1036.0,839.0){\rule[-0.200pt]{0.400pt}{4.818pt}}
\put(1238.0,131.0){\rule[-0.200pt]{0.400pt}{4.818pt}}
\put(1238,90){\makebox(0,0){ 2500}}
\put(1238.0,839.0){\rule[-0.200pt]{0.400pt}{4.818pt}}
\put(1439.0,131.0){\rule[-0.200pt]{0.400pt}{4.818pt}}
\put(1439,90){\makebox(0,0){ 3000}}
\put(1439.0,839.0){\rule[-0.200pt]{0.400pt}{4.818pt}}
\put(231.0,131.0){\rule[-0.200pt]{0.400pt}{175.375pt}}
\put(231.0,131.0){\rule[-0.200pt]{291.007pt}{0.400pt}}
\put(1439.0,131.0){\rule[-0.200pt]{0.400pt}{175.375pt}}
\put(231.0,859.0){\rule[-0.200pt]{291.007pt}{0.400pt}}
\put(30,495){\makebox(0,0){seconds}}
\put(835,29){\makebox(0,0){N = elements in a row}}
\put(1279,819){\makebox(0,0)[r]{4 threads}}
\put(1299.0,819.0){\rule[-0.200pt]{24.090pt}{0.400pt}}
\put(235,131){\usebox{\plotpoint}}
\multiput(271.00,131.58)(18.894,0.491){17}{\rule{14.620pt}{0.118pt}}
\multiput(271.00,130.17)(332.655,10.000){2}{\rule{7.310pt}{0.400pt}}
\multiput(634.00,141.58)(2.490,0.499){159}{\rule{2.085pt}{0.120pt}}
\multiput(634.00,140.17)(397.672,81.000){2}{\rule{1.043pt}{0.400pt}}
\multiput(1036.00,222.58)(1.039,0.500){385}{\rule{0.931pt}{0.120pt}}
\multiput(1036.00,221.17)(401.068,194.000){2}{\rule{0.465pt}{0.400pt}}
\put(235,131){\makebox(0,0){$+$}}
\put(271,131){\makebox(0,0){$+$}}
\put(634,141){\makebox(0,0){$+$}}
\put(1036,222){\makebox(0,0){$+$}}
\put(1439,416){\makebox(0,0){$+$}}
\put(1349,819){\makebox(0,0){$+$}}
\put(235.0,131.0){\rule[-0.200pt]{8.672pt}{0.400pt}}
\put(1279,778){\makebox(0,0)[r]{serial}}
\multiput(1299,778)(20.756,0.000){5}{\usebox{\plotpoint}}
\put(1399,778){\usebox{\plotpoint}}
\put(235,131){\usebox{\plotpoint}}
\multiput(235,131)(20.756,0.000){2}{\usebox{\plotpoint}}
\multiput(271,131)(20.710,1.369){18}{\usebox{\plotpoint}}
\multiput(634,155)(19.301,7.634){21}{\usebox{\plotpoint}}
\multiput(1036,314)(13.168,16.043){30}{\usebox{\plotpoint}}
\put(1439,805){\usebox{\plotpoint}}
\put(235,131){\makebox(0,0){$\times$}}
\put(271,131){\makebox(0,0){$\times$}}
\put(634,155){\makebox(0,0){$\times$}}
\put(1036,314){\makebox(0,0){$\times$}}
\put(1439,805){\makebox(0,0){$\times$}}
\put(1349,778){\makebox(0,0){$\times$}}
\put(231.0,131.0){\rule[-0.200pt]{0.400pt}{175.375pt}}
\put(231.0,131.0){\rule[-0.200pt]{291.007pt}{0.400pt}}
\put(1439.0,131.0){\rule[-0.200pt]{0.400pt}{175.375pt}}
\put(231.0,859.0){\rule[-0.200pt]{291.007pt}{0.400pt}}
\end{picture}

		\caption{seconds to complete multiplication on an NxN matrix }
		\label{runtimes}
\end{figure}


\begin{figure}[p]
		\centering
		% GNUPLOT: LaTeX picture
\setlength{\unitlength}{0.240900pt}
\ifx\plotpoint\undefined\newsavebox{\plotpoint}\fi
\sbox{\plotpoint}{\rule[-0.200pt]{0.400pt}{0.400pt}}%
\begin{picture}(1500,900)(0,0)
\sbox{\plotpoint}{\rule[-0.200pt]{0.400pt}{0.400pt}}%
\put(231.0,131.0){\rule[-0.200pt]{4.818pt}{0.400pt}}
\put(211,131){\makebox(0,0)[r]{ 2}}
\put(1419.0,131.0){\rule[-0.200pt]{4.818pt}{0.400pt}}
\put(231.0,204.0){\rule[-0.200pt]{4.818pt}{0.400pt}}
\put(211,204){\makebox(0,0)[r]{ 2.5}}
\put(1419.0,204.0){\rule[-0.200pt]{4.818pt}{0.400pt}}
\put(231.0,277.0){\rule[-0.200pt]{4.818pt}{0.400pt}}
\put(211,277){\makebox(0,0)[r]{ 3}}
\put(1419.0,277.0){\rule[-0.200pt]{4.818pt}{0.400pt}}
\put(231.0,349.0){\rule[-0.200pt]{4.818pt}{0.400pt}}
\put(211,349){\makebox(0,0)[r]{ 3.5}}
\put(1419.0,349.0){\rule[-0.200pt]{4.818pt}{0.400pt}}
\put(231.0,422.0){\rule[-0.200pt]{4.818pt}{0.400pt}}
\put(211,422){\makebox(0,0)[r]{ 4}}
\put(1419.0,422.0){\rule[-0.200pt]{4.818pt}{0.400pt}}
\put(231.0,495.0){\rule[-0.200pt]{4.818pt}{0.400pt}}
\put(211,495){\makebox(0,0)[r]{ 4.5}}
\put(1419.0,495.0){\rule[-0.200pt]{4.818pt}{0.400pt}}
\put(231.0,568.0){\rule[-0.200pt]{4.818pt}{0.400pt}}
\put(211,568){\makebox(0,0)[r]{ 5}}
\put(1419.0,568.0){\rule[-0.200pt]{4.818pt}{0.400pt}}
\put(231.0,641.0){\rule[-0.200pt]{4.818pt}{0.400pt}}
\put(211,641){\makebox(0,0)[r]{ 5.5}}
\put(1419.0,641.0){\rule[-0.200pt]{4.818pt}{0.400pt}}
\put(231.0,713.0){\rule[-0.200pt]{4.818pt}{0.400pt}}
\put(211,713){\makebox(0,0)[r]{ 6}}
\put(1419.0,713.0){\rule[-0.200pt]{4.818pt}{0.400pt}}
\put(231.0,786.0){\rule[-0.200pt]{4.818pt}{0.400pt}}
\put(211,786){\makebox(0,0)[r]{ 6.5}}
\put(1419.0,786.0){\rule[-0.200pt]{4.818pt}{0.400pt}}
\put(231.0,859.0){\rule[-0.200pt]{4.818pt}{0.400pt}}
\put(211,859){\makebox(0,0)[r]{ 7}}
\put(1419.0,859.0){\rule[-0.200pt]{4.818pt}{0.400pt}}
\put(231.0,131.0){\rule[-0.200pt]{0.400pt}{4.818pt}}
\put(231,90){\makebox(0,0){ 1}}
\put(231.0,839.0){\rule[-0.200pt]{0.400pt}{4.818pt}}
\put(365.0,131.0){\rule[-0.200pt]{0.400pt}{4.818pt}}
\put(365,90){\makebox(0,0){ 2}}
\put(365.0,839.0){\rule[-0.200pt]{0.400pt}{4.818pt}}
\put(499.0,131.0){\rule[-0.200pt]{0.400pt}{4.818pt}}
\put(499,90){\makebox(0,0){ 3}}
\put(499.0,839.0){\rule[-0.200pt]{0.400pt}{4.818pt}}
\put(634.0,131.0){\rule[-0.200pt]{0.400pt}{4.818pt}}
\put(634,90){\makebox(0,0){ 4}}
\put(634.0,839.0){\rule[-0.200pt]{0.400pt}{4.818pt}}
\put(768.0,131.0){\rule[-0.200pt]{0.400pt}{4.818pt}}
\put(768,90){\makebox(0,0){ 5}}
\put(768.0,839.0){\rule[-0.200pt]{0.400pt}{4.818pt}}
\put(902.0,131.0){\rule[-0.200pt]{0.400pt}{4.818pt}}
\put(902,90){\makebox(0,0){ 6}}
\put(902.0,839.0){\rule[-0.200pt]{0.400pt}{4.818pt}}
\put(1036.0,131.0){\rule[-0.200pt]{0.400pt}{4.818pt}}
\put(1036,90){\makebox(0,0){ 7}}
\put(1036.0,839.0){\rule[-0.200pt]{0.400pt}{4.818pt}}
\put(1171.0,131.0){\rule[-0.200pt]{0.400pt}{4.818pt}}
\put(1171,90){\makebox(0,0){ 8}}
\put(1171.0,839.0){\rule[-0.200pt]{0.400pt}{4.818pt}}
\put(1305.0,131.0){\rule[-0.200pt]{0.400pt}{4.818pt}}
\put(1305,90){\makebox(0,0){ 9}}
\put(1305.0,839.0){\rule[-0.200pt]{0.400pt}{4.818pt}}
\put(1439.0,131.0){\rule[-0.200pt]{0.400pt}{4.818pt}}
\put(1439,90){\makebox(0,0){ 10}}
\put(1439.0,839.0){\rule[-0.200pt]{0.400pt}{4.818pt}}
\put(231.0,131.0){\rule[-0.200pt]{0.400pt}{175.375pt}}
\put(231.0,131.0){\rule[-0.200pt]{291.007pt}{0.400pt}}
\put(1439.0,131.0){\rule[-0.200pt]{0.400pt}{175.375pt}}
\put(231.0,859.0){\rule[-0.200pt]{291.007pt}{0.400pt}}
\put(30,495){\makebox(0,0){seconds}}
\put(835,29){\makebox(0,0){num of threads}}
\put(231,829){\usebox{\plotpoint}}
\multiput(231.58,824.51)(0.499,-1.229){265}{\rule{0.120pt}{1.082pt}}
\multiput(230.17,826.75)(134.000,-326.754){2}{\rule{0.400pt}{0.541pt}}
\multiput(365.58,497.27)(0.499,-0.698){265}{\rule{0.120pt}{0.658pt}}
\multiput(364.17,498.63)(134.000,-185.634){2}{\rule{0.400pt}{0.329pt}}
\multiput(499.00,311.92)(0.643,-0.499){207}{\rule{0.614pt}{0.120pt}}
\multiput(499.00,312.17)(133.725,-105.000){2}{\rule{0.307pt}{0.400pt}}
\multiput(634.00,206.93)(14.848,-0.477){7}{\rule{10.820pt}{0.115pt}}
\multiput(634.00,207.17)(111.543,-5.000){2}{\rule{5.410pt}{0.400pt}}
\put(768,201.17){\rule{26.900pt}{0.400pt}}
\multiput(768.00,202.17)(78.168,-2.000){2}{\rule{13.450pt}{0.400pt}}
\put(902,199.67){\rule{32.281pt}{0.400pt}}
\multiput(902.00,200.17)(67.000,-1.000){2}{\rule{16.140pt}{0.400pt}}
\multiput(1036.00,200.59)(14.959,0.477){7}{\rule{10.900pt}{0.115pt}}
\multiput(1036.00,199.17)(112.377,5.000){2}{\rule{5.450pt}{0.400pt}}
\multiput(1171.00,203.95)(29.709,-0.447){3}{\rule{17.967pt}{0.108pt}}
\multiput(1171.00,204.17)(96.709,-3.000){2}{\rule{8.983pt}{0.400pt}}
\multiput(1305.00,202.60)(19.490,0.468){5}{\rule{13.500pt}{0.113pt}}
\multiput(1305.00,201.17)(105.980,4.000){2}{\rule{6.750pt}{0.400pt}}
\put(231,829){\makebox(0,0){$+$}}
\put(365,500){\makebox(0,0){$+$}}
\put(499,313){\makebox(0,0){$+$}}
\put(634,208){\makebox(0,0){$+$}}
\put(768,203){\makebox(0,0){$+$}}
\put(902,201){\makebox(0,0){$+$}}
\put(1036,200){\makebox(0,0){$+$}}
\put(1171,205){\makebox(0,0){$+$}}
\put(1305,202){\makebox(0,0){$+$}}
\put(1439,206){\makebox(0,0){$+$}}
\put(231.0,131.0){\rule[-0.200pt]{0.400pt}{175.375pt}}
\put(231.0,131.0){\rule[-0.200pt]{291.007pt}{0.400pt}}
\put(1439.0,131.0){\rule[-0.200pt]{0.400pt}{175.375pt}}
\put(231.0,859.0){\rule[-0.200pt]{291.007pt}{0.400pt}}
\end{picture}

		\caption{seconds to complete multiplication on an 1000x1000
		matrix}
		\label{runtimes}
\end{figure}

\end{document}
